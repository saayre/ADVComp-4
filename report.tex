\documentclass[12pt]{article}

\usepackage{fullpage}
\usepackage{graphicx}
\usepackage{caption}
\usepackage{subcaption}
\usepackage{float}

\begin{document}


\title{Diffusion Limited Aggregation}
\date{}
\author{Sayre Christenson, Phys 6960}

\maketitle


\begin{abstract}

Diffusion limited aggregation generates a graph of adjoining points.


\end{abstract}

\section*{Introduction}

The diffusion limited aggregation algorithm is a rough computational approximation of certain physical processes like grain growth in cooling liquid metals and dielectric breakdown.
The algorithm random walks a particle along a grid until it touches another particle, then leaves that particle and starts a new one.
Initial layouts can be specified, and those lead to visual differences in the final fractal pattern, along with the quantitative difference of fractal dimension.

This implementation of diffusion limited aggregation has the following steps:

\begin{enumerate}
\item{Place the initial pattern, or seed.}
\item{Randomly place a particle on the edge of a 2-D grid (with side of 512) with toroidal periodic boundary conditions.}
\item{Random walk the particle until it touches an already placed particle.  (See \ref{Aggregration Runs} for more details.)}
\item{Repeat the previous two steps for 10000 particles.}
\item{Measure the dimension of the resulting pattern.  (See \ref{Dimensionality} for more details.)}
\end{enumerate}


\section*{Aggregation}

There are two interpretations of the ending condition of each individual random walk:

\begin{enumerate}
\item{A particle occupies a space horizontally or vertically adjacent grid point to the updated position of the walking particle.}
\item{A particle occupies a space diagonally, horizontally, or vertically adjacent grid point to the updated position of the walking particle.}
\end{enumerate}

The choice of interpretation does affect the features of the fractal (as shown in Figure \ref{hdv}).
Horizontal and vertical only aggregations are much more densely packed, and so they grow outward much less quickly than diagonal aggregations.
Also, for the single point seed, whether or not diagonal touching is allowed generally determines the direction of the larger branches.
This is an effect of the ``square'' nature of the particles used, because there are more ways a particle can move to diagonal touching than horizontal or vertical touching.

%% hdv
\begin{figure}[h]
  \centering

  \begin{subfigure}[b]{0.45\textwidth}
    \fbox{\includegraphics[width=1.1\textwidth]{hvmap.png}}
    \caption{Horizontal and vertical aggregation only.}
  \end{subfigure}
  \qquad
  \begin{subfigure}[b]{0.45\textwidth}
    \fbox{\includegraphics[width=1.1\textwidth]{dmap.png}}
    \caption{Including diagonal aggregation.}
  \end{subfigure}

  \caption{Side = 512, Number of Particles = 10000}
  \label{hdv}

\end{figure}

In the previous figures, a single particle was placed at approximately the center of the grid as the growth seed.
Any starting pattern representable as a set of 2-D coordinates is possible, and in particular a circle is tested in \ref{Extra Experimentation}.


\section*{Dimensionality}


The method for computing the dimension of a fractal image relies on the equation

\[
D = \frac{\log(N)}{\log(1/s)} .
\]

The way $N$ and $s$ are counted is as follows: the image is split up into boxes of size $s$ and the number of boxes with a non-zero pixel inside is $N$.
Repeating this process yields several data points, and a linear regression returns the slope, $D$, the fractal's dimension.
From experimentation, the line fit is straightest when considering only box sizes less than 100 pixels, so the chosen box sizes for finding the dimensionality of each fractal is the set of even numbers less than or equal to 100.

The dimensionality for the same images used in Figure \ref{hdv} are -1.555 and -1.539, as shown in Figure \ref{dims}.
Even though the images are easily distinguishable by the naked eye, the specific rules on how particles touch do not seem to have an effect of more than a few percent on the fractal dimension.

%% dims
\begin{figure}
  \centering

  \begin{subfigure}[b]{0.45\textwidth}
    \includegraphics[width=1.2\textwidth]{hvdims.png}
  \end{subfigure}
  \qquad
  \begin{subfigure}[b]{0.45\textwidth}
    \includegraphics[width=1.2\textwidth]{ddims.png}
  \end{subfigure}

  \caption{Opposite slope is caused by using the logarithm of the box size rather than the inverse box size.}
  \label{dims}

\end{figure}


\section*{Extra Experimentation}

%% circle start
Several of the example slides had seeds with sharp geometric features, and demonstrated that sharp features were more likely to aggregate particles than flat features.
However, a quick follow-up question is: is there any consistent direction particles will aggregate on when the seed has no sharp features?
The images in Figure \ref{circle} were produced by seeding all grid points less than a specified distance from the center.
%% talk about results

%% circle, whole page?
%% \begin{figure}
%%   \centering

%%   \begin{subfigure}[b]{0.45\textwidth}

%% top edge only start
Another question easily answered is: to what extent does anisotropy affect the process of aggregation?  
Phrased differently, this question asks how the fractal pattern changes when particles only approach from certain directions.
This is accomplished within this implementation by only placing particles on certain image edges to start, and selectively removing periodic boundary conditions.


\section*{Conclusions}


\end{document}
