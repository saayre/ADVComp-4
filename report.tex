\documentclass[12pt]{article}

\usepackage{fullpage}


\begin{document}


\title{Diffusion Limited Aggregation}
\date{}
\author{Sayre Christenson}

\maketitle


\begin{abstract}

Diffusion limited aggregation generates a graph of adjoining points.


\end{abstract}

\section*{Introduction}

The diffusion limited aggregation algorithm is a rough computational approximation of certain physical processes like grain growth in cooling liquid metals and dielectric breakdown.
The algorithm random walks a particle along a grid until it touches another particle, then leaves that particle and starts a new one.
Initial layouts can be specified, and those lead to qualitative differences in the final fractal pattern.


\section*{Aggregation Runs}

There are two interpretations of the ending condition of each individual random walk:

\begin{enumerate}
\item{A particle occupies a space horizontally or vertically adjacent grid point to the updated position of the walking particle.}
\item{A particle occupies a space diagonally, horizontally, or vertically adjacent grid point to the updated position of the walking particle.}
\end{enumerate}

Little difference can be observed between each at the large scale, though at the smallest scale the difference between horizontal/vertical and diagonal branching are apparent.



\section*{Dimensionality}


\section*{Conclusions}
