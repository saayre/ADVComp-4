\documentclass[12pt]{article}

\usepackage{fullpage}


\begin{document}


\title{Diffusion Limited Aggregation}
\date{}
\author{Sayre Christenson}

\maketitle


\begin{abstract}

Diffusion limited aggregation generates a graph of adjoining points.


\end{abstract}

\section*{Introduction}

The diffusion limited aggregation algorithm is a rough computational approximation of certain physical processes like grain growth in cooling liquid metals and dielectric breakdown.
The algorithm random walks a particle along a grid until it touches another particle, then leaves that particle and starts a new one.
Initial layouts can be specified, and those lead to visual differences in the final fractal pattern, along with the quantitative difference of fractal dimension.

This implementation of diffusion limited aggregation has the following steps:

\begin{enumerate}
\item{Place the initial pattern.}
\item{Randomly place a particle on the edge of a 2-D grid (with side of 512) with toroidal periodic boundary conditions.}
\item{Random walk the particle until it touches an already placed particle.  (See \ref{Aggregration Runs} for more details.)}
\item{Repeat the previous two steps for 10000 particles.}
\item{Measure the dimension of the resulting pattern.  (See \ref{Dimensionality} for more details.)}
\end{enumerate}


\section*{Aggregation Runs}

There are two interpretations of the ending condition of each individual random walk:

\begin{enumerate}
\item{A particle occupies a space horizontally or vertically adjacent grid point to the updated position of the walking particle.}
\item{A particle occupies a space diagonally, horizontally, or vertically adjacent grid point to the updated position of the walking particle.}
\end{enumerate}

Little difference can be observed (as in Figure 1) between each at the large scale, though at the smallest scale the difference between horizontal/vertical and diagonal branching are apparent.

\includegraphics{}


In the previous figure, a single particle was placed at approximately the center of the grid as the growth seed.
Any starting pattern representable as a set of 2-D coordinates is possible, and some are tested in \ref{Extre Experimentation}.


\section*{Dimensionality}


The definition of dimension is

\[
D 


\section*{Extra Experimentation}

%% circle start

%% top edge only start


\section*{Conclusions}
